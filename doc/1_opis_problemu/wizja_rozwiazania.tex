\section{Wizja rozwiązania}

\subsection{Wstęp}

\hspace{0.5cm} Celem produktu, który chcemy dostarczyć, jest zautomatyzowanie zapytań do różnych serwerów DNS, weryfikacja ich poprawności, oraz w razie wykrycia błędnie zwracanego IP, jak najszybsze powiadomienie użytkownika. Aplikacja ma także działać dla wielu użytkowników jednocześnie. Domyślnie użytkownikiem mogłaby stać się dowolna osoba, która posiada własny serwis internetowy i po wykupieniu subskrypcji byłaby chroniona. 

\subsection{Back-end}

\hspace{0.5cm} Pierwszą część naszego projektu stanowiłby program działający na serwerze, który miałby odpytywać cyklicznie serwery DNS na całym świecie o podane przez klientów strony internetowe. W przypadku stwierdzenia nieprawidłowości byłby on w stanie natychmiast poinformować klienta gdzie wystąpił problem tak, aby ten zdołał jak najszybciej zareagować. Mógłby on współpracować z dowolnym VPNem, podanym przez administratora.

\subsection{Front-end}

\hspace{0.5cm} Dla użytkowników udostępniona zostałaby strona internetowa, która umożliwiałaby im interakcję z nasza aplikacją. Nowi użytkownicy musieliby się najpierw zarejestrować, po czym mieliby możliwość wykupienie subskrypcji. Subskrypcja dotyczyłaby jednego lub kilku serwisów, które byłyby regularnie sprawdzane.

Po ponownym zalogowaniu byłaby dla nich dostępna możliwość zarządzania subskrypcjami, wykupienia dodatkowych, oraz sprawdzenia statystyk. Statystyki dotyczyłyby testów wykonanych dla podanych przez nich serwisów.

Na tej samej stronie mógłby się także zalogować administrator systemu, który miałby dostęp do ustawień oraz listy klientów i wykupionych przez nich subskrypcji.